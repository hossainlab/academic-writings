%================ch3======================================
\chapter{Experiments}\label{ch:ch3}

\section{First section of this chapter}
An experiment is a procedure carried out to support, refute, or validate a hypothesis. Experiments provide insight into cause-and-effect by demonstrating what outcome occurs when a particular factor is manipulated. Experiments vary greatly in goal and scale, but always rely on repeatable procedure and logical analysis of the results. There also exists natural experimental studies.

\subsection{First subsection of this chapter}
A child may carry out basic experiments to understand gravity, while teams of scientists may take years of systematic investigation to advance their understanding of a phenomenon. Experiments and other types of hands-on activities are very important to student learning in the science classroom. Experiments can raise test scores and help a student become more engaged and interested in the material they are learning, especially when used over time. Experiments can vary from personal and informal natural comparisons (e.g. tasting a range of chocolates to find a favorite), to highly controlled (e.g. tests requiring complex apparatus overseen by many scientists that hope to discover information about subatomic particles). Uses of experiments vary considerably between the natural and human sciences.

An experiment usually tests a hypothesis, which is an expectation about how a particular process or phenomenon works. However, an experiment may also aim to answer a ``what-if'' question, without a specific expectation about what the experiment reveals, or to confirm prior results. If an experiment is carefully conducted, the results usually either support or disprove the hypothesis. According to some philosophies of science, an experiment can never "prove" a hypothesis, it can only add support. On the other hand, an experiment that provides a counterexample can disprove a theory or hypothesis, but a theory can always be salvaged by appropriate ad hoc modifications at the expense of simplicity. An experiment must also control the possible confounding factors—any factors that would mar the accuracy or repeatability of the experiment or the ability to interpret the results. Confounding is commonly eliminated through scientific controls and/or, in randomized experiments, through random assignment.

\subsubsection{First subsubsection of this chapter}
In engineering and the physical sciences, experiments are a primary component of the scientific method. They are used to test theories and hypotheses about how physical processes work under particular conditions (e.g., whether a particular engineering process can produce a desired chemical compound). Typically, experiments in these fields focus on replication of identical procedures in hopes of producing identical results in each replication. Random assignment is uncommon.

\paragraph{First paragraph of this chapter.}
According to his explanation, a strictly controlled test execution with a sensibility for the subjectivity and susceptibility of outcomes due to the nature of man is necessary.

Examples of citation in brackets \citep{wd} and without brackets~\citet{ch}.  














