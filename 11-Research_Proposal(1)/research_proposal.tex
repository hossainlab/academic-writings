%%%%%%%%%%%%%Article template by C Has%%%%%%%%%%%%%%%%%%%%%%%%%%%%%%%%%%%%%
\documentclass[10pt, a4paper]{article}
\usepackage[T1]{fontenc}
\usepackage{times}
\usepackage{amsmath, amssymb, amsfonts}   
\usepackage{graphicx}
\usepackage{caption, subcaption}
\usepackage[margin=2cm]{geometry}
\usepackage[onehalfspacing]{setspace}
\usepackage[round, sort & compress]{natbib}
\usepackage[hidelinks]{hyperref}
\title{\vspace{-15mm} Health Data Research Organization, Dhaka, Bangladesh \\ Title of Your Research}
\author{Md. Jubayer Hossain\\ b150605021@mib.jnu.ac.bd}
\date{}

\begin{document}
\maketitle

\section*{Background -- Justification [\textit{This section should have linked references}]}
\begin{itemize}
	\item Review in a first bullet the global consequences of the health problem being examined in terms of death, disability, effectiveness and cost-effective of interventions \cite{hossain2020lack}. Avoid general statements and provide quantified data when available. Follow by explaining how this problem effects the place where the research is being considered.
	\item Provide information on completed, ongoing or planned prevention/control efforts targeting this problem ins South Asia, India, and/or the State/Place where the research will be conducted.
	\item Specify the information needs of different stakeholders(health professionals/pub health programme policy/policy makers) to improve this health problem why currently available information is insufficient.
\end{itemize}



\section*{Objectives [\textit{Describe primary and secondary objectives of your research}]}
\begin{itemize}
	\item Need to be stated quantitatively for the primary outcome(\textit{Make it clear whether you propose to estimate a quantity or whether you propose to test a hypothesis})
	\item Clearly distinguish secondary from the primary objectives 
\end{itemize}

\section*{Methods [\textit{Describe your research methodology}]} 
Resources for specific requirements for different studies -- \url{https://www.equator-network.org/}

\subsection*{Study Population}
\begin{itemize}
	\item Specify the population in which will undertake the study
\end{itemize}
\subsection*{Study Design}
\begin{itemize}
	\item Describe the type of study(e.g; survey, case-control, cohort studies) in one short bullet.
\end{itemize}
\subsection*{Operational definitions}
\begin{itemize}
	\item Provide information regarding the key destinations, criteria and / or control requirement strategy that you will be using. 
\end{itemize}

\subsection*{Sampling Procedure}
\begin{itemize}
	\item Describe the type of sampling techniques(probability/non-probability) you will be using.  
\end{itemize}

\subsection*{Sample Size}
\begin{itemize}
	\item Briefly mention your sample size and the assumptions you used to calculate it. This should contain enough information for the reader to redo the calculations to check the estimate.  
\end{itemize}
\subsection*{Data Collection Procedure}
\begin{itemize}
	\item Explain shortly who will collect what kind of data, what the timeline is and what quality assurance mechanism will be used.
\end{itemize}

\subsection*{Human Participant Protection}
\begin{itemize}
	\item Mention key measures taken to ensure the protection of human participants in your study and which ethics committee will review the proposal. 
\end{itemize}


\subsection*{Expected Benefits}
\begin{itemize}
	\item Describe the expected output(e.g; reports) that this study will generate and the timeline
	\item Describe the expected outcome: How this study will influence management of this problem in question in the area where the research will be conducted.
\end{itemize}

\bibliographystyle{plainnat}
\bibliography{references}

\end{document}












